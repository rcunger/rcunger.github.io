\documentclass[compress,usenames,dvipsnames,8pt]{beamer}
\usepackage{amsmath,amsthm,thmtools}
\usepackage{mathtools}
\usepackage{amssymb}
%% General document %%%%%%%%%%%%%%%%%%%%%%%%%%%%%%%%%%
\usepackage{graphicx}
\usepackage{tikz-cd}
\usepackage{mathrsfs}  
\usepackage{subfig}
\usepackage{slashed}
\usepackage{appendixnumberbeamer}
\usepackage{transparent}
\usepackage{stackengine}
\usepackage{relsize}
\usepackage{bm}
\usepackage{multirow}
%\usepackage{enumitem}



\usepackage[style=verbose-ibid,backend=biber]{biblatex}
\bibliography{scattering_interior} 
\usetikzlibrary{decorations.fractals}
%%%%%%%%%%%%%%%%%%%%%%%%%%%%%%%%%%%%%%%%%%%%%%%%%%%%%%
\usepackage{multimedia}

%% Beamer Layout %%%%%%%%%%%%%%%%%%%%%%%%%%%%%%%%%%
\useoutertheme[subsection=false,shadow]{miniframes}
\useinnertheme{default}
\usefonttheme{default}
%\usepackage{palatino}

\usepackage{caption}
\captionsetup[figure]{labelformat=empty}% redefines the caption setup of the figures environment in the beamer class.


\newtheorem*{thm}{Theorem}
\newtheorem*{prop}{Proposition}
\newtheorem{cor}{Corollary}
\newtheorem*{theorem1}{Theorem 1}
\newtheorem*{theorem2}{Theorem}
\newtheorem*{lem}{Lemma}
\newtheorem*{conjecture}{Conjecture}
\newtheorem*{conjecture_nonum}{Conjecture}
%\setbeamerfont{title like}{shape=\scshape}
%\setbeamerfont{frametitle}{shape=\scshape}


\theoremstyle{definition}
\newtheorem*{q}{Question}

\setbeamercolor{frametitle}{fg=black}
\setbeamercolor*{lower separation line head}{bg=DeepSkyBlue4} 
\setbeamercolor*{normal text}{fg=black,bg=white} 
\setbeamercolor*{alerted text}{fg=red} 
\setbeamercolor*{example text}{fg=black} 
\setbeamercolor*{structure}{fg=black} 
 
\setbeamercolor*{palette tertiary}{fg=black,bg=black!10} 
\setbeamercolor*{palette quaternary}{fg=black,bg=black!10} 

\setbeamertemplate{theorems}[ams style] 

\setbeamertemplate{section in toc}[sections numbered]
\renewcommand{\d}{\mathrm{d}}
\newcommand{\Hp}{\mathcal{H}}
\newcommand{\Ho}{\mathcal{H}}
\newcommand{\Ch}{\mathcal{CH}}

\renewcommand{\(}{\begin{columns}}
\renewcommand{\)}{\end{columns}}
\newcommand{\<}[1]{\begin{column}{#1}}
\renewcommand{\>}{\end{column}}
%%%%%%%%%%%%%%%%%%%%%%%%%%%%%%%%%%%%%%%%%%%%%%%%%%
\newcommand{\executeiffilenewer}[3]{%
	\ifnum\pdfstrcmp{\pdffilemoddate{#1}}%
	{\pdffilemoddate{#2}}>0%
	{\immediate\write18{#3}}\fi%
}
\newcommand{\includesvg}[1]{%
	\executeiffilenewer{#1.svg}{#1.pdf}%
	{inkscape -z -D --file=#1.svg %
		--export-pdf=#1.pdf --export-latex}%
	\input{#1.pdf_tex}%
}
\makeatletter
\newcommand\mathcircled[1]{%
	\mathpalette\@mathcircled{#1}%
}
\newcommand\@mathcircled[2]{%
	\tikz[baseline=(math.base)] \node[draw=red,circle,inner sep=.5pt] (math) {$\m@th#1#2$};%
}
\makeatother
\makeatletter
\newcommand\mathcircledd[1]{%
	\mathpalette\@mathcircledd{#1}%
}
\newcommand\@mathcircledd[2]{%
	\tikz[baseline=(math.base)] \node[draw=blue,circle,inner sep=.5pt] (math) {$\m@th#1#2$};%
}
\makeatother
\beamertemplatenavigationsymbolsempty
\setbeamertemplate{headline}{}
\setbeamertemplate{footline}[frame number]
\makeatletter
\def\blfootnote{\xdef\@thefnmark{}\@footnotetext}
\makeatother
%\input{automatic_penrose.tex}
\usepackage{colortbl}
\begin{document}
\setbeamertemplate{caption}{\raggedright\insertcaption\par}


%%%%%%%%%%%%%%%%%%%%%%%%%%%%%%%%%%%%%%%%%%%%%%%%%%%%%%
%%%%%%%%%%%%%%%%%%%%%%%%%%%%%%%%%%%%%%%%%%%%%%%%%%%%%%
%\section*{\scshape Introduction}
\begin{frame}
\title{Retiring the third law of black hole thermodynamics}
%\subtitle{~\\}
\author{Ryan Unger}
\institute{\small 
Princeton University  %\\ Institute for Theoretical Studies
}
\date{
\small{ Z\"urich PDE \& mathematical physics  seminar }\\
December 2022 \\~\\
%joint work with Christoph Kehle
}
\titlepage
\end{frame}


\begin{frame}{Classical thermodynamics}


\pause

\begin{center}
\begin{tabular}{|cc|}
\hline
Law & Thermodynamics \\
\hline \hline 
& \\
Zeroth & $T$ constant throughout body in thermal equilibrium\\ 
& \\ 
 \hline 
& \\
 &   \\
& \\
\hline 
& \\
&  \\
& \\
\hline
& \\
 &  \\
& \\
\hline
\end{tabular}
\end{center}



\end{frame}


\begin{frame}{Classical thermodynamics}


\begin{center}
\begin{tabular}{|cc|}
\hline
Law & Thermodynamics \\
\hline \hline 
& \\
Zeroth & $T$ constant throughout body in thermal equilibrium\\ 
& \\ 
 \hline 
& \\
 First & $dE=TdS+\text{work terms}\,(-PdV)$  \\
& \\
\hline 
& \\
& \\
& \\
\hline
& \\
& \\
& \\
\hline
\end{tabular}
\end{center}

\addtocounter{framenumber}{-1}

\end{frame}



\begin{frame}{Classical thermodynamics}


\begin{center}
\begin{tabular}{|cc|}
\hline
Law & Thermodynamics \\
\hline \hline 
& \\
Zeroth & $T$ constant throughout body in thermal equilibrium\\ 
& \\ 
 \hline 
& \\
 First & $dE=TdS+\text{work terms}\,(PdV)$  \\
& \\
\hline 
& \\
Second & $\delta S\ge 0$ in any process \\
& \\
\hline
& \\
& \\
& \\
\hline
\end{tabular}
\end{center}
\addtocounter{framenumber}{-1}

\end{frame}



\begin{frame}{Classical thermodynamics}


\begin{center}
\begin{tabular}{|cc|}
\hline
Law & Thermodynamics \\
\hline \hline 
& \\
Zeroth & $T$ constant throughout body in thermal equilibrium\\ 
& \\ 
 \hline 
& \\
 First & $dE=TdS+\text{work terms}\,(PdV)$  \\
& \\
\hline 
& \\
Second & $\delta S\ge 0$ in any process \\
& \\
\hline
& \\
Third & Impossible to achieve $T=0$ in finite time \\
& \\
\hline
\end{tabular}
\end{center}

\addtocounter{framenumber}{-1}


\end{frame}


\begin{frame}{Black hole thermodynamics}

\begin{figure}
				\includegraphics[scale=0.3]{BCH-2.png}
			\end{figure}

\begin{center}
\textbf{Black hole thermodynamics} is a proposed close mathematical \textbf{analogy} between black hole dynamics and classical thermodynamics. 
\end{center}

\pause 

\begin{itemize}
\item What is a black hole?
\item What is the analogue of temperature $T$?
\item What is the analogue of entropy $S$?
\end{itemize}
\end{frame}

\begin{frame}{Crash course on black holes: Spacetimes and Einstein's equations}

The setting is Einstein's theory of general relativity. A \textbf{spacetime} consists of a 4-manifold $\mathcal M^{3+1}$ and a Lorentzian metric $g$ satisfying the \textbf{Einstein field equations}
\begin{equation}
\tag{EFE} Ric(g)-\tfrac 12 R(g)g = 2T.
\end{equation}
\pause

~\\ 

\textbf{Example: Minkowski space.} $\mathcal M^{3+1}= \mathbb R^{3+1}_{t,x,y,z}$ and 
\[g=-dt^2 + dx^2 +dy^2 +dz^2\]
This metric describes the geometry of special relativity with speed of light $c=1$. 
\pause

~\\ 

Given a vector $v\in T_p\mathcal M$: 
\begin{itemize}
\item $g(v,v)<0$, $v$ is \emph{timelike}
\item $g(v,v)=0$, $v$ is \emph{null}
\item $g(v,v)> 0$, $v$ is \emph{spacelike}
\end{itemize}
Curves with null or timelike tangent vector let us define \textbf{causality} on a spacetime. 

\end{frame}



\begin{frame}{Crash course on black holes: Geometry of Minkowski space}
\begin{figure}
				\includegraphics[width=\textwidth]{aretakis-notes.png}
			\end{figure}
\end{frame}



\begin{frame}{Crash course on black holes: What is a black hole?}

\textbf{Asymptotic flatness:}  $g$ approaches the Minkowski metric ``at infinity." \pause

~\\ 

An asymptotically flat spacetime is said to contain a \textbf{black hole region} \[\mathcal{BH}\subset\mathcal M\]
if observers and signals originating in $\mathcal{BH}$ cannot reach infinity. \pause

~\\ 

The \textbf{event horizon} of the black hole is its topological boundary
\[\mathcal H^+ := \partial(\mathcal{BH}).\] 


~\\

\pause The event horizon $\mathcal{H}^+$ is a \textbf{null hypersurface}: it is ruled by geodesics with null tangent vectors. \pause





\end{frame}



\begin{frame}{Crash course on black holes: The Cauchy problem}

\begin{itemize}

\item Dynamics in general relativity is properly phrased in terms of the \emph{Cauchy problem}
\item \emph{Cauchy data}: induced (Riemannian) geometry on a spacelike hypersurface $(\Sigma^3,g_0,k_0)$ together with initial data for matter fields \pause
\end{itemize}

\begin{thm}[Choquet-Bruhat '52, Choquet-Bruhat--Geroch '69]
Any Cauchy data $(\Sigma^3,g_0,k_0,matter)$ for the Einstein equations coupled to a suitable matter model admits a unique maximal globally hyperbolic development which solves the Einstein field equations
\[Ric(g)-\tfrac 12 R(g)g = 2T.\]
\end{thm}
 

\end{frame}



\begin{frame}{Crash course on black holes: Penrose diagrams}

\textbf{Penrose diagrams:} A $(1+1)$-dimensional diagrammatic representation of spacetimes. This is particularly useful for visualizing causality. \pause

~\\ 

\begin{itemize}
\item $45$ degree lines correspond to null hypersurfaces in $(3+1)$ dimensions
\item steeper than 45: timelike
\item shallower than 45: spacelike
\end{itemize}



~\\

\begin{figure}
 \def\svgwidth{8pc}
\input{Mink-penrose.pdf_tex}
\caption{Penrose diagram of Minkowski space}
\end{figure}

\end{frame}



\begin{frame}{Crash course on black holes: Minkowski space}

\begin{figure}
 \def\svgwidth{9pc}
\input{Mink-penrose-3.pdf_tex}
\end{figure}

\end{frame}

\begin{frame}{Crash course on black holes: Minkowski space}
\addtocounter{framenumber}{-1}
\begin{figure}
 \def\svgwidth{9pc}
\input{Mink-penrose-2.pdf_tex}
\end{figure}

\end{frame}

\begin{frame}{Crash course on black holes: Minkowski space}
\addtocounter{framenumber}{-1}
\begin{figure}
 \def\svgwidth{9pc}
\input{Mink-penrose-4.pdf_tex}
\end{figure}

\end{frame}

\begin{frame}{Crash course on black holes: Minkowski space}
\addtocounter{framenumber}{-1}
\begin{figure}
 \def\svgwidth{9pc}
\input{Mink-penrose-5.pdf_tex}
\end{figure}

\end{frame}

\begin{frame}{Crash course on black holes: The Schwarzschild solution}

\begin{itemize}
\item Mass $M>0$ 
\[g_M= -\left(1-\frac{2M}{r}\right)dt^2 +\left(1-\frac{2M}{r}\right)^{-1}dr^2 + r^2(d\theta^2+\sin^2\theta\,d\phi^2)\]
\item Solves the \textbf{Einstein vacuum} equations
\item Describes a static, nonrotating black hole
\end{itemize}
\pause

\begin{figure}
 \def\svgwidth{20pc}
\input{Schw-penrose.pdf_tex}
\end{figure}



\end{frame}


\begin{frame}{Crash course on black holes: The Schwarzschild solution}

\begin{itemize}
\item Mass $M>0$ 
\[g_M= -\left(1-\frac{2M}{r}\right)dt^2 +\left(1-\frac{2M}{r}\right)^{-1}dr^2 + r^2(d\theta^2+\sin^2\theta\,d\phi^2)\]
\item Solves the \textbf{Einstein vacuum} equations
\item Describes a static, nonrotating black hole
\end{itemize}
\begin{figure}
 \def\svgwidth{20pc}
\input{Schw-penrose-2.pdf_tex}
\end{figure}

\addtocounter{framenumber}{-1}



\end{frame}

\begin{frame}{Crash course on black holes: The Schwarzschild solution}

\begin{itemize}
\item Mass $M>0$ 
\[g_M= -\left(1-\frac{2M}{r}\right)dt^2 +\left(1-\frac{2M}{r}\right)^{-1}dr^2 + r^2(d\theta^2+\sin^2\theta\,d\phi^2)\]
\item Solves the \textbf{Einstein vacuum} equations
\item Describes a static, nonrotating black hole
\end{itemize}
\begin{figure}
 \def\svgwidth{20pc}
\input{Schw-penrose-3.pdf_tex}
\end{figure}

\addtocounter{framenumber}{-1}

\end{frame}

\begin{frame}{Crash course on black holes: Reissner--Nordstr\"om}

\begin{itemize}
\item Mass $M>0$ and charge $0\le |e|\le M$
\[g_{M,e}=-\left(1-\frac{2M}{r}+\frac{e^2}{r^2}\right)dt^2+\left(1-\frac{2M}{r}+\frac{e^2}{r^2}\right)^{-1}dr^2+r^2(d\theta^2+\sin^2\theta\,d\phi^2)\]

\item Solves the \textbf{Einstein-Maxwell} equations

\item Describes a static, charged, nonrotating black hole 

\item $e=0$: Schwarzschild 
\end{itemize}
\pause

~\\

\begin{columns}%\hspace{1cm}
		\begin{column}{0.4\linewidth}
			\begin{figure}
 \def\svgwidth{10pc}
\input{RN-penrose.pdf_tex}
\caption{\textbf{Subextremal} $|e|<M$}
\end{figure}
		\end{column}
		\begin{column}{0.4\linewidth}
		\pause	\begin{figure}
 \def\svgwidth{10pc}
\input{ERN-penrose.pdf_tex}
\caption{\textbf{Extremal} $|e|=M$}
\end{figure}
		\end{column}
	\end{columns}
	\pause
	\begin{center}
	\textbf{Superextremal} $|e|>M$. Does not contain a black hole!
	\end{center}

\end{frame}

%\begin{frame}{Crash course on black holes: Kerr and Kerr--Newmann}

%Kerr:
%\begin{itemize}
%\item Mass $M>0$ and specific angular momentum $0\le |a|\le M$
%\item Solves the \textbf{Einstein vacuum} equations
%\item Describes a stationary, rotating black hole \pause 
%\item $|a|< M$ \textbf{subextremal}, $|a|=M$ \textbf{extremal}\pause
%\end{itemize}

%~\\

%Kerr--Newmann
%\begin{itemize}
%\item Combination of Reissner--Nordstr\"om and Kerr solutions
%\item Has mass, charge, and angular momentum
%\end{itemize}

%\end{frame}


\begin{frame}{Crash course on black holes: Surface gravity and extremality}

\begin{itemize}
\item The event horizons in these exact solutions are \textbf{Killing horizons}: The spacetime admits a Killing field which is null and tangent to the horizon \pause

\item Associated to a Killing horizon is a function $\kappa\ge 0$, called the \textbf{surface gravity} \pause

\item Physically, $\kappa$ measures the force required to hold a test mass ``at the horizon" ``from infinity" \pause

%\item It is also related to the celebrated \textbf{(local) redshift effect} \pause

\item For Reissner--Nordstr\"om, 
\[\kappa(g_{M,e})=\frac{\sqrt{M^2-e^2}}{(M+\sqrt{M^2-e^2})^2}\] \pause

\item \textbf{Extremal:} $\kappa = 0$ ($|e|=M$ Reissner--Nordstr\"om)

\item \textbf{Subextremal:} $\kappa>0$ ($|e|<M$, including Schwarzschild) \pause

\item The celebrated \textbf{Kerr} family also has subextremal and extremal members
\end{itemize}

\end{frame} 



\begin{frame}{Black hole thermodynamics}


\pause

\begin{center}
\begin{tabular}{|ccc|}
\hline
Law & Classical thermodynamics & Black holes \\
\hline \hline 
& & \\
Zeroth &  $T$ constant in equilibrium & \onslide<3->{surface gravity $\kappa$ constant on stationary horizon} \\ 
& & \\ 
\hline 
& & \\
\onslide<4->{First} & \onslide<4->{$dE=TdS+\cdots$} & \onslide<5->{$dM=\kappa dA+\cdots$} \\
& & \\
\hline 
& & \\
\onslide<6->{Second} & \onslide<6->{$\delta S\ge 0$} & \onslide<7->{$dA\ge 0$} \\
& & \\
\hline
& & \\
\onslide<8->{Third} & \onslide<8->{$T\not\to0$ in finite time} & \onslide<9->{\textcolor{magenta}{surface gravity $\kappa\not\to 0$ in finite time}} \\
& & \\
\hline
\end{tabular}
\end{center}

~\\

\begin{itemize}
\item \onslide<10->{Proposed by Bardeen--Carter--Hawking '73}
\item \onslide<11->{Hawking radiation---interpret $\kappa\propto T$}
\item \onslide<12->{Bekenstein entropy---interpret $A\propto S$}
\item \onslide<13->{Laws 0, 1, and 2 proved (when suitably interpreted!) by Hawking, Carter, Bardeen--Carter--Hawking, Wald} 
\end{itemize}

\end{frame}

\begin{frame}{Retiring the third law}

\begin{center}
Original formulation of Bardeen--Carter--Hawking:
\end{center}

\begin{figure}
				\includegraphics[scale=0.3]{third-law-BCH.png}
			\end{figure}

%~\\

%\pause

%\begin{center}
%Statement revised by Israel:
%\end{center}




%\begin{figure}
%				\includegraphics[scale=0.3]{Israel-2.png}
%			\end{figure}


\end{frame}


\begin{frame}{Retiring the third law}



\begin{conjecture}[The third law, BCH '73, \textcolor{blue}{Israel '86}]
 A subextremal black hole cannot become extremal in finite time by any continuous process, no matter how idealized, in which the spacetime and matter fields remain \textcolor{blue}{regular} and obey the \textcolor{blue}{weak energy condition}. 
\end{conjecture}

\pause

~\\

\begin{thm}[Kehle--U. '22]
Subextremal black holes can become extremal in finite time, evolving from \underline{regular} initial data. In particular, {\underline{the ``third law of black hole thermodynamics'' is false}}.
\end{thm}

~\\

\pause 

More precisely, there exist \underline{regular} solutions of the Einstein--Maxwell-charged scalar field system with the following behavior:


\end{frame}

% In fact, there exist regular one-ended Cauchy data for the Einstein--Maxwell-charged scalar field system which undergo gravitational collapse and form an exactly Schwarzschild apparent horizon, only for the spacetime to form an exactly extremal Reissner--Nordstr\"om event horizon at a later advanced time. 

\begin{frame}{Retiring the third law}

\begin{figure}
 \def\svgwidth{15pc}
\input{first-page-revised.pdf_tex}
\end{figure}

\begin{itemize}
\item Spherically symmetric Cauchy data for the Einstein--Maxwell-charged scalar field system which undergo gravitational collapse \pause

\item Forms an exactly Schwarzschild ``apparent horizon"

\item  Forms an exactly extremal Reissner--Nordstr\"om event horizon at a later advanced time \pause

\item The solution is arbitrarily regular: for any $k\in\mathbb N$, there exists a $C^k$ example
\end{itemize}


\end{frame}

\begin{frame}{The strategy}

\textbf{Steps in the proof:} \pause

\begin{enumerate}
\item Construct spacetime teleologically \pause
\begin{itemize}
\item Teleological construction allows us to locate the event horizon \pause
\end{itemize}
\item Extract a Cauchy hypersurface $\Sigma$ and deduce that the teleologically constructed spacetime arises \emph{dynamically} from initial data
\end{enumerate}

~\\ \pause

\begin{columns}%\hspace{1cm}
		\begin{column}{0.2\textwidth}
			\begin{figure}
				\includegraphics[width=\textwidth]{Aristotle.jpg}
			\end{figure}
		\end{column}
		\begin{column}{0.8\textwidth}
			\begin{center}
					The black hole region $\mathcal{BH}$ and event horizon $\mathcal{H}^+$ of a spacetime are \textbf{teleological} notions---they can't be known to exist, much less be located, \emph{without knowing the entire future of the spacetime}. 	\\
				%	~\\
				%	\pause The event horizon $\mathcal{H^+}$ is not a singular object. 
						\end{center}
		\end{column}
	\end{columns}


\end{frame}



\begin{frame}{Einstein--Maxwell-charged scalar field system}

\begin{itemize}
\item  Lorentzian manifold $(\mathcal M^{3+1},g)$
\item 2-form $F=dA$ (electromagnetism) 
\item Charged (complex) scalar field $\phi$ \pause
\item Equations: \begin{align}
    R_{\mu\nu}(g)-\tfrac 12 R(g)g_{\mu\nu}&= 2\left(  T^\mathrm{EM}_{\mu\nu}+  T^\mathrm{CSF}_{\mu\nu}\right)\\
    \nabla^\mu F_{\mu\nu}&= 2\mathfrak e\, \mathrm{Im}(\phi\overline{D_\nu \phi})\\
    g^{\mu\nu}D_\mu D_\nu \phi&=0\\
    T^\mathrm{EM}_{\mu \nu}&= g^{\alpha \beta}F _{\alpha \nu}F_{\beta \mu }-\tfrac{1}{4}F^{\alpha \beta}F_{\alpha \beta}g_{\mu \nu}\\
  T^\mathrm{CSF}_{\mu \nu}&= \mathrm{Re}(D _{\mu}\phi \overline{D _{\nu}\phi}) -\tfrac{1}{2}g_{\mu \nu} g^{\alpha \beta} D _{\alpha}\phi \overline{D _{\beta}\phi}
\end{align}
\item In an appropriate gauge, this is a nonlinear hyperbolic system for $(g,F,\phi)$ \pause

\item Spherical symmetry!
\end{itemize}

\end{frame}




\begin{frame}{Prototype: Minkowski to Schwarzschild gluing}

The type of Penrose diagram we want:

%\addtocounter{framenumber}{-1}
\begin{figure}
 \def\svgwidth{20pc}
\input{char-gluing-3.pdf_tex}
\end{figure}

We want to {\bf{glue}} solutions of the Einstein-scalar field system

\end{frame}

\begin{frame}{Prototype: Minkowski to Schwarzschild gluing}

The type of Penrose diagram we want:

\addtocounter{framenumber}{-1}
\begin{figure}
 \def\svgwidth{20pc}
\input{char-gluing-6.pdf_tex}
\end{figure}

We want to {\bf{glue}} solutions of the Einstein-scalar field system


\end{frame}


\begin{frame}{Toy model: gluing for the linear wave equation}

We now fix a spacetime $(\mathcal M^{3+1},g)$ and consider the \textbf{linear wave equation}
\begin{equation}
\Box_g \phi=0\tag{$*$}
\end{equation}

\pause

\begin{q}
Given two domains $\mathfrak R_1,\mathfrak R_2\subset \mathcal M$ and functions 
\begin{align*}
\phi_1:\mathfrak R_1\to \mathbb R\\
\phi_2:\mathfrak R_2\to \mathbb R
\end{align*}
solving $(*)$\pause, when does there exist a function 
\[\phi:\mathcal M\to \mathbb R\] solving $(*)$, such that 
\begin{align*}
\phi|_{\mathfrak R_1}&= \phi_1\\
\phi|_{\mathfrak R_2} &= \phi_2\,?
\end{align*}
\end{q}

\end{frame}

\begin{frame}{Spacelike gluing for the linear wave equation}

Consider first the case of two spacelike separated regions:

\begin{figure}
 \def\svgwidth{20pc}
\input{spacelike-gluing.pdf_tex}
\end{figure}

~\\

\pause

In this case the gluing problem for $(*)$ is trivial:

\begin{prop}[Cauchy problem for the wave equation]
Given a spacelike hypersurface $\Sigma\subset (\mathcal M,g)$ and functions $\phi_0,\phi_1\in C^\infty(\Sigma)$, there exists a unique function $\phi:\mathcal M\to \mathbb R$ solving 
\[\Box_g\phi =0 \tag{$*$}\]
such that $\phi|_\Sigma=\phi_0$, $n(\phi)|_\Sigma = \phi_1$.
\end{prop}

~\\

\pause

Analogue for the Einstein equations is a highly nontrivial problem in elliptic PDE! (\footnotesize{Corvino--Schoen, Chru\'{s}ciel--Delay, Li--Yu, Carlotto--Schoen, Li--Mei, Hintz, ...})

\end{frame}

\begin{frame}{Characteristic/null gluing for the linear wave equation}

%To achieve the Penrose diagram of the third law violation, we need to solve a gluing problem along a null hypersurface:

\begin{figure}
 \def\svgwidth{10pc}
\input{char-gluing-1.pdf_tex}
\end{figure}

\pause
Analogy:
\begin{itemize}
\item $\mathfrak R_1$ is Minkowski
\item $\mathfrak R_2$ is Schwarzschild
\item $C$ will be the event horizon $\mathcal H^+$
\end{itemize}

\end{frame}

\begin{frame}{The characteristic initial value problem}

\textbf{Bifurcate null hypersurface:} $C\cup\underline C$ where $C$ and $\underline C$ meet transversally, $C\cap \underline C \approx S^2$ \pause
\begin{prop}[Characteristic IVP for the wave equation]
Given a bifurcate null hypersurface $C\cup\underline C$ and $\phi_0:C\cup \underline C\to\mathbb R$, there exists a unique function $\phi$ defined to the future of $C\cup \underline C$ such that 
\[\Box_g \phi =0 \tag{$*$}\]
and $\phi|_{C\cup \underline C}= \phi_0$. 
\end{prop}

\pause

\begin{figure}
 \def\svgwidth{10pc}
\input{char-IVP-1 copy.pdf_tex}
\end{figure}



\begin{center}
\textcolor{white}{\underline{Characteristic data for the wave equation does not involve derivatives.}}
\end{center}
\end{frame}

\begin{frame}{The characteristic initial value problem}

\textbf{Bifurcate null hypersurface:} $C\cup\underline C$ where $C$ and $\underline C$ meet transversally, $C\cap \underline C \approx S^2$ 
\begin{prop}[Characteristic IVP for the wave equation]
Given a bifurcate null hypersurface $C\cup\underline C$ and $\phi_0:C\cup \underline C\to\mathbb R$, there exists a unique function $\phi$ defined to the future of $C\cup \underline C$ such that 
\[\Box_g \phi =0 \tag{$*$}\]
and $\phi|_{C\cup \underline C}= \phi_0$. 
\end{prop}

\addtocounter{framenumber}{-1}

\begin{figure}
 \def\svgwidth{10pc}
\input{char-IVP-1.pdf_tex}
\end{figure}



\begin{center}
\textcolor{white}{\underline{Characteristic data for the wave equation does not involve derivatives.}}
\end{center}
\end{frame}

\begin{frame}{The characteristic initial value problem}

\textbf{Bifurcate null hypersurface:} $C\cup\underline C$ where $C$ and $\underline C$ meet transversally, $C\cap \underline C \approx S^2$ 
\begin{prop}[Characteristic IVP for the wave equation]
Given a bifurcate null hypersurface $C\cup\underline C$ and $\phi_0:C\cup \underline C\to\mathbb R$, there exists a unique function $\phi$ defined to the future of $C\cup \underline C$ such that 
\[\Box_g \phi =0 \tag{$*$}\]
and $\phi|_{C\cup \underline C}= \phi_0$. 
\end{prop}

\addtocounter{framenumber}{-1}

\begin{figure}
 \def\svgwidth{10pc}
\input{char-IVP-1.pdf_tex}
\end{figure}



\begin{center}
\underline{Characteristic data for the wave equation does not involve derivatives.}
\end{center}
\end{frame}


\begin{frame}{The null ``constraints"}

In any well posed initial value problem, there needs to be a mechanism to compute the full solution on the initial data hypersurface. \pause Example:

~\\

\begin{itemize}
\item Wave equation on  Minkowski $\mathbb R^{3+1}$ 

\item $u=t-r$, $v=t+r$ double null coordinates \pause

\item $\phi=\phi(t,r,\theta^1,\theta^2)=\phi(u,v,\theta^1,\theta^2)$ satisfies 
\begin{equation}
\partial_u \partial_v (r\phi)=\slashed\Delta(r\phi)\tag{$*$}
\end{equation}
\item \pause Integrate $(*)$ on $C$:
\[\partial_u(r\phi)|_{v_2}-\partial_u(r\phi)|_{v_1}=\int_{v_1}^{v_2} \slashed\Delta(r\phi)\,dv'\]
\item \pause $\Longrightarrow$ $\partial_u\phi|_C$ can be calculated from (1) $\partial_u \phi$ at the bifurcation sphere and (2) $\phi|_C$\pause 

\item $(*)$ lets us compute $\phi$ to arbitrarily high order on $C$

\end{itemize}

\end{frame}

\begin{frame}{The null ``constraints"}

\begin{center}
In general, the wave equation can be interpreted as a transport equation for the transverse derivative of $\phi$, sourced by $\phi$ and tangential derivatives (which are a part of the characteristic data)
\end{center} \pause

~\\

\[\partial_v\left(\partial_u \phi\right)+(known)\partial_u\phi=data\]

~\\

\pause Commuting the wave equation with transverse derivatives gives an inductive system of transport equations for all orders!

~\\

\pause These equations have become known as the \emph{null ``constraints"} and are very distinct from the ``usual" constraint equations for Einstein. 
\pause \footnotesize{(The Einstein equations in double null gauge also have true constraints which are more reminiscent of the spacelike constraints.)}

\end{frame}

\begin{frame}{Back to characteristic gluing }


\begin{figure}
 \def\svgwidth{10pc}
\input{char-gluing-2.pdf_tex}
\end{figure}

The $C^k$ characteristic gluing problem reduces to: \pause

\begin{itemize}
\item Given two spheres $S_1$ and $S_2$ along an outgoing null cone $C$
\item Given $k$ ingoing and outgoing derivatives of $\phi$ at $S_1$ and $S_2$ \pause
\item Prescribe $\phi$ along the part of $C$ between $S_1$ and $S_2$ so that the outgoing derivatives agree with the given ones and the solutions of the transport equations for the ingoing derivatives have the specified initial and final values. \pause
\item Initial/boundary value problem for a system of coupled ODEs where you are allowed to (almost) freely prescribe the potential
\end{itemize}

\end{frame}

\begin{frame}{Previous work on characteristic gluing}

Complete $C^1$ theory for the linear wave equation obtained by \textbf{Aretakis} '17 
\begin{itemize}
\item Finitely many obstructions: conserved charges \pause
\item Conserved charges arise from \textbf{conservation laws} along $C$ ({\footnotesize{Generalizing the conservation law responsible for the \emph{Aretakis instability}}})\pause
\end{itemize}

~\\ 

Perturbative characteristic gluing for the Einstein vacuum equations by \textbf{Aretakis--Czimek--Rodnianski} '21
\begin{itemize}
\item Gluing for two data sets close to Minkowski space \pause
\item Linearize the Einstein equations about Minkowski, develop procedure to glue the linearized equations ({\footnotesize{Dafermos--Holzegel--Rodnianski}})
\item Inverse function theorem \pause
\item Conserved charges arise from kernel of linearized equations and form 10-dimensional obstruction space \pause
\item \textbf{Czimek--Rodnianski} '22: obstruction free characteristic gluing by adding high-frequency perturbations 
\end{itemize}

\end{frame}



\begin{frame}{Minkowski to Schwarzschild gluing}

\begin{figure}
 \def\svgwidth{20pc}
\input{char-gluing-3.pdf_tex}
\end{figure}

\end{frame}

\begin{frame}{Minkowski to Schwarzschild gluing}

\addtocounter{framenumber}{-1}
\begin{figure}
 \def\svgwidth{20pc}
\input{char-gluing-5.pdf_tex}
\end{figure}

\end{frame}

\begin{frame}{Minkowski to Schwarzschild gluing}

\addtocounter{framenumber}{-1}
\begin{figure}
 \def\svgwidth{20pc}
\input{char-gluing-3.pdf_tex}
\end{figure}

\end{frame}




\begin{frame}{Einstein-scalar field in spherical symmetry}

\begin{itemize}
\item $\mathcal M^{3+1}=\mathcal Q^{1+1}\times S^2$ 
\[g=-\Omega^2 du \,dv+r^2 g_{S^2}\] 
\item $\Omega(u,v)>0$ lapse, $r(u,v)>0$ area-radius  \pause
\item Wave equations 
\begin{align}\label{eq:CSF-wave-equation}
\partial_u \partial_v \phi &= - \frac{\partial_u\phi\partial_v r}{r} - \frac{\partial_u r\partial_v\phi}{r}  \\
\label{eq:CSF-equation-for-r}
\partial_u \partial_v r &= -\frac{\Omega^2}{4r} - \frac{\partial_ur \partial_vr}{r}  \\
\label{eq:CSF-equation-for-Omega}
 \partial_u\partial_v\log(\Omega^2)&= \frac{\Omega^2}{2r^2}+2\frac{\partial_u r\partial_v r}{r^2}  
\end{align}\pause
\item Raychaudhuri's equations (constraints)
\begin{align}
\partial_u\left(\frac{\partial_u r}{\Omega^2}\right)&=-\frac{r}{\Omega^2}(\partial_u\phi)^2 \label{eq:CSF-Raychaudhuri-u}\\
\label{eq:CSF-Raychaudhuri-v}
\partial_v\left(\frac{\partial_v r}{\Omega^2}\right)&=-\frac{r}{\Omega^2}(\partial_v\phi)^2
\end{align}
\end{itemize}

\end{frame}

\begin{frame}{Minkowski to Schwarzschild gluing}
\begin{thm}[Kehle--U. '22]
For any $k\in \mathbb N$ and $0<R_i<2M_f$, the Minkowski sphere of radius $R_i$ can be characteristically glued to the Schwarzschild event horizon sphere with mass $M_f$ to order $C^k$ within the Einstein-scalar field model in spherical symmetry. 
\end{thm}

\begin{figure}
 \def\svgwidth{14pc}
\input{three-bumps.pdf_tex}
\end{figure}

\end{frame}

\begin{frame}{Idea of the proof}
\begin{itemize}
\item Null constraint system becomes a coupled nonlinear system of ODEs sourced by $\phi(v)$ \pause

\item Scalar field ansatz \[\phi_\alpha(v) = \sum_{1\le j\le k+1}\alpha_j \chi_j(v),\quad \alpha\in \mathbb R^{k+1}\] \pause

\item Initialize $\partial_u \phi_\alpha(0)=\cdots = \partial_u^k\phi_\alpha(0)=0$ \pause

\item Initialize $r(v)$ teleologically at $v=1$ ($\partial_v r(0)$ is a gauge choice, but $\partial_v r(1)=0$!) \pause

\item For each $\alpha$, generate $r(v)$, $\partial_u r(v)$, $\partial_u\phi_\alpha(v)$, etc. \pause

\item The antipodal map $\alpha\mapsto -\alpha$ leaves geometric quantities fixed ($r$, $\Omega^2$, and derivatives) \pause

\item The antipodal map $\alpha\mapsto -\alpha$ changes the sign of the scalar field and its derivatives \pause

\item In particular, the map 
\[\alpha\mapsto \left(\partial_u \phi_\alpha(1),\dotsc, \partial_u^k\phi_\alpha(1)\right)\] \pause
is odd

\item \textbf{Borsuk--Ulam theorem:} there exists $\alpha_*$ such that 
\[ \left(\partial_u \phi_{\alpha_*}(1),\dotsc, \partial_u^k\phi_{\alpha_*}(1)\right)=0.\]
\end{itemize}

\end{frame}



\begin{frame}{Disproof of the third law}
\begin{figure}
 \def\svgwidth{24pc}
\input{trapped-surface-proof copy.pdf_tex}
\end{figure}
\end{frame}

\begin{frame}{Future directions}

\begin{itemize}
\item Critical behavior: Reissner--Nordstr\"om with $|e|>M$ does not contain a black hole. \pause Can we embed our solutions into a 1-parameter family of spacetimes exhibiting this behavior? \pause

\item Event horizon gluing for the Einstein vacuum equations: 

\begin{conjecture}
There exist Cauchy data for the  Einstein vacuum equations
\begin{equation*}
    Ric(g) =0
\end{equation*}
which undergo gravitational collapse and form an exactly Schwarzschild apparent horizon, only for the spacetime to form an exactly extremal Kerr event horizon at a later advanced time. 
In particular,  \underline{already in vacuum},    the ``third law of black hole thermodynamics'' is false.
\end{conjecture}

~\\ 

~\\ 

\pause

\begin{center}
\emph{Thank you!}
\end{center}

\end{itemize}

\end{frame}




\end{document}
\begin{frame}{Idea of the proof II}

The proof exploits three key ideas:

\begin{itemize}
\item Initialize $r(v)$ teleologically, i.e., at $v=1$. All other quantities are initialized at $v=0$. 

\item Properly exploiting gauge freedom reduces the number of quantities that must actually be glued

\item The even/odd structure of the Einstein-scalar field system
\end{itemize}


\end{frame}


\begin{frame}
\begin{tabular}{ |p{3cm}|p{3cm}|p{3cm}|  }
\hline
\multicolumn{3}{|c|}{Country List} \\
\hline
Country Name or Area Name& ISO ALPHA 2 Code &ISO ALPHA 3 \\
\hline
Afghanistan & AF &AFG \\
Aland Islands & AX   & ALA \\
Albania &AL & ALB \\
Algeria    &DZ & DZA \\
American Samoa & AS & ASM \\
Andorra & AD & AND   \\
Angola & AO & AGO \\
\hline
\end{tabular}

\end{frame



\begin{frame}{Conservation laws and characteristic gluing}

\begin{thm}[Aretakis '17]
Given a null hypersurface $C\subset (\mathcal M,g)$ and $S_1, S_2\subset C$, and $1$-jets prescribed at $S_1$ and $S_2$, the $C^1$ characteristic gluing problem can be solved in the sense of the previous slide provided that finitely many {\bf{conserved charges}} evaluated at $S_1$ and $S_2$ are equal.
\end{thm}

\pause

\begin{itemize}
\item Conserved charges arise from \textbf{conservation laws} along $C$ ({\footnotesize{Generalizing the conservation law responsible for the \emph{Aretakis instability}}}) \pause
\item Charges are related to the kernel of an elliptic operator defined on a foliation of $C$, hence finitely many linearly independent charges
\end{itemize}

\end{frame}





\begin{frame}{Perturbative characteristic gluing for the Einstein equations}

\begin{itemize}
\item Define a notion of sphere data for the Einstein equations (data induced on an $S^2\subset\mathcal M$)

\item Define a notion of data along a null hypersurface (foliated by $S^2$'s) \pause

\item Linearize the Einstein equations about Minkowski, develop procedure to glue the linearized equations (\footnotesize{Dafermos--Holzegel--Rodnianski})
\end{itemize}
\pause

\begin{thm}[Aretakis--Czimek--Rodnianski '21]
Two sphere data sets for the Einstein equations which are $\varepsilon$-close to spheres in Minkowski of radii $1$ and $2$, can be glued along a null hypersurface to $C^2$ order (up to gauge transformations), as a solution of the Einstein vacuum null constraint system, provided a certain set of 10 (linearly) conserved charges are equal on the two spheres.
\end{thm}\pause

\begin{thm}[Czimek--Rodnianski '22]
By taking advantage of the precise nonlinear structure of the Einstein equations, gluing can be achieved even if the linearly conserved charges are not equal. 
\end{thm} 
\pause

\begin{itemize}
\item Einstein vacuum equations in double null gauge are very complicated \pause

\item Free data (shear $\hat\chi$) is geometrically difficult to work with \pause

\item Unclear how to go beyond the perturbative regime \pause \begin{itemize}
\item Linearization allows gluing to be done step by step, exploiting the ``nilpotent" structure of the constraints. Nonlinear gluing has to be done in one shot!
\end{itemize}
\end{itemize}

\end{frame}




\begin{frame}{Crash course on black holes: Teleology}



\begin{columns}%\hspace{1cm}
		\begin{column}{0.2\textwidth}
			\begin{figure}
				\includegraphics[width=\textwidth]{Aristotle.jpg}
			\end{figure}
		\end{column}
		\begin{column}{0.8\textwidth}
			\begin{center}
					The black hole region $\mathcal{BH}$ and event horizon $\mathcal{H}^+$ of a spacetime are \textbf{teleological} notions---they can't be known to exist, much less be located, \emph{without knowing the entire future of the spacetime}. 	\\
					~\\
					\pause The event horizon $\mathcal{H^+}$ is not a singular object. 
						\end{center}
		\end{column}
	\end{columns}
	
~\\ 

~\\

\pause The formation, dynamics, and stability of black holes can be rigorously studied using the tools of Lorentzian geometry and hyperbolic PDE. 

\end{frame}


